\chapter*{Wstęp}

Współczesne organizacje funkcjonują w środowisku, w którym skuteczne zarządzanie dokumentami staje się kluczowe dla zapewnienia efektywności pracy oraz przestrzegania przepisów prawnych. W miarę jak firmy rozwijają się i generują coraz większe ilości dokumentów, konieczne staje się wdrożenie nowoczesnych narzędzi, które pozwolą na sprawne zarządzanie obiegiem informacji. Tradycyjne metody obiegu dokumentów, oparte na papierze, są czasochłonne, podatne na błędy i opóźnienia, a także ograniczają dostępność danych. W związku z tym, coraz więcej firm decyduje się na wdrożenie systemów informatycznych, które usprawniają zarządzanie dokumentami, umożliwiając łatwe ich tworzenie, edytowanie, archiwizowanie i usuwanie.

Projekt, którego celem jest stworzenie systemu informatycznego do zarządzania dokumentami w firmie, odpowiada na rosnącą potrzebę automatyzacji tych procesów. System ten ma na celu uproszczenie obiegu dokumentów, zapewniając jednocześnie pełną kontrolę nad ich tworzeniem, przechowywaniem oraz dostępem. Dzięki wdrożeniu nowoczesnego systemu, pracownicy będą mogli szybko i bezpiecznie zarządzać dokumentami, a firma będzie mogła skutecznie przestrzegać obowiązujących regulacji prawnych dotyczących przechowywania i archiwizowania dokumentów.

Aplikacja pozwoli na łatwe tworzenie, edytowanie oraz archiwizowanie dokumentów, zapewniając przejrzystość w obiegu informacji w firmie. Dodatkowo, system umożliwi tworzenie kategorii dokumentów, co ułatwi organizację materiałów i szybki dostęp do niezbędnych informacji. Ważnym aspektem będzie również funkcjonalność związana z usuwaniem dokumentów, umożliwiająca ich bezpieczne usunięcie, zgodnie z wymaganiami prawnymi dotyczącymi przechowywania danych.
