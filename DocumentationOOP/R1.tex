% ********** Rozdział 1 **********
\chapter{Opis założeń projektu}

\section{Cele projektu}
Celem projektu jest stworzenie systemu informatycznego wspierającego zarządzanie dokumentami w firmie. System ma na celu usprawnienie obiegu dokumentów, umożliwiając ich łatwe tworzenie, edytowanie, archiwizowanie oraz usuwanie. Zaimplementowane funkcje mają zapewnić pełną kontrolę nad dokumentami w firmie, umożliwiając pracownikom szybki dostęp do niezbędnych materiałów, a także ich bezpieczne przechowywanie.

System ma zapewnić możliwość tworzenia, edytowania, usuwania dokumentów oraz ich archiwizowania, umożliwiając pracownikom szybsze zarządzanie dokumentacją, jak również zapewnienie zgodności z procedurami firmowymi dotyczącymi obiegu dokumentów.

\section{Wymagania funkcjonalne i niefunkcjonalne}

\noindent \textbf{Wymagania funkcjonalne}

\begin{itemize}
    \item System musi umożliwiać tworzenie, edytowanie, archiwizowanie oraz usuwanie dokumentów.
    \item Użytkownicy powinni mieć możliwość przeglądania szczegółowych informacji o dokumentach, takich jak data utworzenia, autor, status oraz historia zmian.
    \item System powinien pozwalać na tworzenie i zarządzanie kategoriami dokumentów, co ułatwi ich klasyfikację oraz szybkie wyszukiwanie.
    \item System powinien umożliwiać przypisywanie uprawnień do dokumentów, aby kontrolować dostęp do wrażliwych materiałów (np. tylko wybrani użytkownicy będą mogli edytować lub usuwać dokumenty).
\end{itemize}

\noindent \textbf{Wymagania niefunkcjonalne}

\begin{itemize}
    \item System musi charakteryzować się wysoką wydajnością, zapewniając płynne przetwarzanie dużych zbiorów dokumentów i umożliwiając szybkie wyszukiwanie.
    \item Powinien być skalowalny, umożliwiając łatwą rozbudowę systemu i dodawanie nowych funkcji w miarę rozwoju firmy.
    \item Interfejs użytkownika powinien być intuicyjny i łatwy w obsłudze, aby zapewnić szybkie wdrożenie pracowników oraz bezproblemowe korzystanie z systemu.
    \item System musi być kompatybilny z popularnymi systemami operacyjnymi.
    \item Należy zapewnić odpowiedni poziom bezpieczeństwa danych, w tym szyfrowanie danych i ochrona przed nieautoryzowanym dostępem.
\end{itemize}

% ********** Koniec rozdziału **********
